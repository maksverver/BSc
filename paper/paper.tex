\documentclass{acm_proc_article-sp}

\usepackage{pslatex}
\usepackage{epsfig}
\usepackage{appendix}
\usepackage{float}
\usepackage{url}

\floatstyle{ruled}
\newfloat{program}{thp}{lop}
\floatname{program}{Program}

\begin{document}

\title{Evaluation of Cache-Oblivious Data Structures}

\numberofauthors{1}
\author{Maks Verver\\ \email{m.verver@student.utwente.nl}}

% Obligatory permission block
\toappear{
Permission to make digital or hard copies of all or part of this work
for personal or classroom use is granted without fee provided that
copies are not made or distributed for profit or commercial advantage
and that copies bear this notice and the full citation on the first
page. To copy otherwise, or republish, to post on servers or to
redistribute to lists, requires prior specific permission.

\textit{\small 9$^{th}$ Twente Student Conference on IT, Enschede, June, 2008}

Copyright 2008, University of Twente, Faculty of Electrical Engineering,
Mathematics and Computer Science}
% End of obligatory permission block

\maketitle

\begin{abstract}
%TODO
\end{abstract}

\keywords{cache efficiency, locality of reference, algorithms}

% TODO: more meaningful section titles?

\section{Introduction}
% Copy and shorten introduction from proposal
% -> Motivate research into cache-efficient algorithms
% -> Move "classification" subsection describing three models to front
% -> Describe each of the three models
% -> Finally, present goal of the research (incorporate "Research Questions" in proposal)

% Marielle suggested:
%  - give the "standard" memory model an explicit name
%  - move example of cache oblivious data structures to front
%  - mention examples of cache unaware and cache aware algorithms

Text here will be mostly copied from the proposal.

\section{Related Work}
% Present related work

Text here will be mostly copied from the proposal.

\section{Use Case}
% TODO: different name?
% Describe use case: state search, set data structure.
% Describe (globally) what to measure: execution time

\section{Research Approach}
% Describe research approach in detail:
% - NIPS VM
% - SPIN model checker
% - Promela-to-NIPS compiler
% - Reimplementation of data structures in C

% - Subsection: B-tree implementation
% - Subsection: Hash-table implementation
% - Subsection: Bender's B-tree implementation

\section{Metrics}
% Describe how performance was measaured
% Describe data set used

\section{Results}
% Present results!
% Tables, graphs, etc.

\section{Conclusion}
% Repeat findings.

\bibliographystyle{plain}
\bibliography{paper}
\end{document}
