% TODO: ensure all references in paper.bib are cited in the paper!
\documentclass{acm_proc_article-sp}

\usepackage{pslatex}
\usepackage{epsfig}
\usepackage{appendix}
\usepackage{float}
\usepackage{url}

\floatstyle{ruled}
\newfloat{program}{thp}{lop}
\floatname{program}{Program}

\begin{document}

\title{Evaluation of Cache-Oblivious Data Structures}
\subtitle{\textit{Work In Progress!}}

\numberofauthors{1}
\author{Maks Verver\\ \email{m.verver@student.utwente.nl}}

% Obligatory permission block
\toappear{
Permission to make digital or hard copies of all or part of this work
for personal or classroom use is granted without fee provided that
copies are not made or distributed for profit or commercial advantage
and that copies bear this notice and the full citation on the first
page. To copy otherwise, or republish, to post on servers or to
redistribute to lists, requires prior specific permission.

\textit{\small 9$^{th}$ Twente Student Conference on IT, Enschede, June, 2008}

Copyright 2008, University of Twente, Faculty of Electrical Engineering,
Mathematics and Computer Science}
% End of obligatory permission block

\maketitle

\begin{abstract}
% TODO: rewrite (NB: max. 10 lines)
%In modern computer hardware architecture, memory is organized in a hierarchy
%consisting of several types of memory with different memory sizes, block
%transfer sizes and access times. The cache-oblivious model has been proposed
%to reflect this reality more accurately than traditional models. A number of
%algorithms and data structures have been proposed that perform optimally in
%this model. The goal of the proposed research project is to implement a
%selection of cache-oblivious set-like data structures, evaluate their
%performance (in terms of execution time and memory use) in a realistic
%environment, and compare their performance to traditional data structures.
\end{abstract}

% TODO: update keywords?
\keywords{cache efficiency, locality of reference, algorithms}

% TODO: more meaningful section titles?

\section{Introduction}
% Copy and shorten introduction from proposal
% -> Motivate research into cache-efficient algorithms
% -> Move "classification" subsection describing three models to front
% -> Describe each of the three models
% -> Introduce three data structures:
%     - note that hash tables have poor locality of reference
%     - note that btrees have good locality of reference but require pagesize
% -> Finally, present goal of the research (incorporate "Research Questions" in
% proposal)

% Marielle suggested:
%  - give the "standard" memory model an explicit name
%  - move example of cache oblivious data structures to front
%  - mention examples of cache unaware and cache aware algorithms

% Note explicitly that cache-oblivious algorithms are relatively complex, so
% that the main question is wether the improvedlocality of reference 
% compensates for the overhead incurred.

Text here will be mostly copied from the proposal.

\section{Related Work}
% Present related work

Text here will be mostly copied from the proposal.

\section{Test Case}
As a test case, the scenario of a state space search has been chosen. This is a
suitable scenario for two reasons. First, it is commonly used as a practical
component of formal methods for software verification, and therefore good
algorithms are of great practical significance. Second, as we will explain
below, the performance of state space search algorithms depends for a large
part on the performance of the data structures that are used to implement
them; therefore, research into efficient data structures is of particular
interest to this application.

State space search is used to verify the correctness of software programs. For
this purpose, programs are first modeled using a formal language, that is also
used to specify properties of the program that should hold during its
execution. A program can be in a (possibly infinite) amount of states, one of
which is usually designated the \emph{initial state}. If a transition from one
state to another is possible (according to the rules of the formal language
used) the latter state is said to be a \emph{successor state} of the former.
Generating the successors of a particular state is also called
\emph{expanding} the state. Of course, the model must be non-deterministic to
allow more than one successor to exist for a single state.

The set of all states reachable by transitions from the initial state is called
the \emph{state space} of a model, and it is the goal of a state space search
algorithm to generate all of these states, in order to check that desired
properties hold in all of them. For this approach it is often required that
the state space be finite (for obvious reasons).

The outline of a state space search algorithm is given in Program
\ref{prog-search}. Note that in addition to the initial state and a function
to generate successors, a queue and a set data structure are used. The queue
holds states that have been generated but not yet expanded and is used to
determine what state to expand next. The set holds all states that have been
generated so far and is used to prevent a single state from being expanded
more than once.

% TODO: note that the algorithm uses the queue, successors and set, and that
the set is the bottleneck.

\begin{program}
\begin{verbatim}
Queue queue = new Queue;
Set generated = new Set;
queue.insert(initial_state);
while (!queue.empty()) {
    State state = queue.extract();
    for (State s : successors(state)) {
        if (generated.contains(s) == false) {
            generated.insert(s);
            queue.insert(s);
        }
    }
}
\end{verbatim}
\caption{Pseudo-code for a simple state search algorithm.}
\label{prog-search}
\end{program}

% TODO: different name?

% Describe case: state search, set data structure.
% Describe (globally) what to measure: execution time

\section{Methodology}

% - NIPS VM
% - SPIN model checker
% - Promela-to-NIPS compiler

For our experimental framework, we need a collection of formal models that are
representative of those typically used for formal verification, and a way to
execute them. For the first part, we have looked at Spin \cite{spin}, a widely
used model checking tool. Spin uses a custom modeling language, called
PROMELA, to specify models and comes with a collection of example models that
are suitable for our experiment.

For the execution of these models we will use NIPS VM \cite{weber2007evm}, a
high-performance virtual machine for state space generation that is easily
embeddable in our framework. Although the NIPS VM only executes bytecode in a
custom format, a PROMELA compiler is also available to generate the required
bytecode from the Spin models.\cite{nips} The NIPS VM is prefered to over the
Spin tools because it was designed to be embedded in other programs and as
such is more easily integrated in our framework.

\section{Data Structures}
% TODO: copy motivation for reimplementation from proposal
% TODO: specify required operations (insert&contains)
% TODO: note that all data structures require contiguous memory
%       (except the Bender set)
% TODO: note use of mmap for automatic storage management

\subsection{Hash Table Implementation}
Hash tables are widely used as a fast (O($1$)) data structure especially when
used for in-memory storage. In its simplest form, a hash table consists of an
index array (the \emph{index}) with a fixed number of slots. A \emph{hash
function} is used to map key values onto slot numbers. If the slot for a value
is empty, it may be inserted there. Queries for the existance of an element in
the hash table similarly see if the queried value is stored at its designated
slot.

Note that if we insert several different values, some values may map to the
same slot, which is problematic if each slot can only store one value. There
are a lot of different ways to resolve this collision problem; we use
\emph{separate chaining} which means that slots do not store values directly,
but instead each slot stores a pointer to a singly-linked list of values that
map to that slot. This particular implementation technique is well-suited to
the scenario where values may have different sizes (as the slots only need to
store a fixed-size pointer and not a variable-size value) and maintains
relatively good performance when the number of values stored exceeds the size
of the index array \cite{sedgewick1998ac}.

% TODO: picture?

Our hash table implementation uses a fixed size index which must be specified
when creating the hash table, as this simplifies the implementation
considerably. The index is stored at the front of the file, after which values
are simply appended in the order in which they are added. Note that we do not
support removing elements from the hash table, which means we do not have to
deal with holes that would otherwise occur in the stored file. For our
experiments the FNV-1a hash function \cite{noll2004fnv} is used (modulo the
size of the index) to map values to slots.

\subsection{B-tree Implementation}
The B-tree data structure was first proposed by Bayer and McCreight
\cite{bayer1970oam} and is widely implemented and often described in
textbooks. Our implementation is based on the description by Kifer, Bernstein
and Lewis \cite{kifer2006dsa}.

B-trees are similar to other search tree data structures. A defining property
is that they organize data in pages of a fixed size, each containing several
values, which makes them especially useful for purposes of on-disk storage
where reading and writing data in a few large blocks is relatively cheap
compared to accessing several smaller blocks.

% TODO: picture and description

Our does not support storing values larger than the size of a page (which
requires special handling, usually by assigning special overflow pages to hold
the portion of the value that does not fit in the page). Deletion of items is
not supported either.

\section{Cache-Oblivious Dictionary Implementation}
% TODO: reference (if not included earlier)

% sparse array
% configurable density parameter
% note that equal division of gaps over all windows is important
%   (and note the problems with the "old" behaviour)
% binary search tree index
% van Emde Boas lay-out
% special optimization on insertion (note that it's not described by Bender)

\subsection{Experimental Environment}

\section{Metrics}
% Describe how performance was measured
% Describe data set used

\section{Results}
% Present results!
% Tables, graphs, etc.
% What to include:
%   per program:     transitions/states graph
%   per program/set: 
%
%   

\section{Discussions}
% Explain results, draw conclusions.

\section{Conclusion}
% Repeat findings.

\bibliographystyle{plain}
\bibliography{paper}
\end{document}
